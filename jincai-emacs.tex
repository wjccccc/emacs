% Created 2020-12-24 四 15:17
% Intended LaTeX compiler: pdflatex
\documentclass[11pt]{article}
\usepackage[utf8]{inputenc}
\usepackage[T1]{fontenc}
\usepackage{graphicx}
\usepackage{grffile}
\usepackage{longtable}
\usepackage{wrapfig}
\usepackage{rotating}
\usepackage[normalem]{ulem}
\usepackage{amsmath}
\usepackage{textcomp}
\usepackage{amssymb}
\usepackage{capt-of}
\usepackage{hyperref}
\author{jincai}
\date{\today}
\title{}
\hypersetup{
 pdfauthor={jincai},
 pdftitle={},
 pdfkeywords={},
 pdfsubject={},
 pdfcreator={Emacs 25.3.1 (Org mode 9.1.4)}, 
 pdflang={English}}
\begin{document}

\tableofcontents

\section{Jincai' Emacs configurations}
\label{sec:org5c4a345}


\subsection{插件管理(Must Top)}
\label{sec:org1750056}

\begin{verbatim}

;; Added by Package.el.  This must come before configurations of
;; installed packages.  Don't delete this line.  If you don't want it,
;; just comment it out by adding a semicolon to the start of the line.
;; You may delete these explanatory comments.
(require 'package)
(add-to-list 'package-archives '("melpa" . "https://melpa.org/packages/") t)
;; Comment/uncomment this line to enable MELPA Stable if desired.  See `package-archive-priorities`
;; and `package-pinned-packages`. Most users will not need or want to do this.
(add-to-list 'package-archives '("melpa-stable" . "http://mirrors.tuna.tsinghua.edu.cn/elpa/melpa/") t)
(setq package-check-signature nil)
(package-initialize)

\end{verbatim}

\subsection{启动初始化}
\label{sec:org3e69e01}

\begin{verbatim}

(if (string-match "XEmacs\\|Lucid" emacs-version)
;;;;;;;;;;;;;;;;;;;;;;;;;;;;;;;;;;;;;;;;;;;;;;;;;;;;;;;;;;;;;;;;;;;;;;;;;
  ;;; XEmacs
  ;;; ------
  ;;;;;;;;;;;;;;;;;;;;;;;;;;;;;;;;;;;;;;;;;;;;;;;;;;;;;;;;;;;;;;;;;;;;;;;;;
  (progn
     (if (file-readable-p "~/.xemacs/init.el")
	(load "~/.xemacs/init.el" nil t))
  )
  ;;;;;;;;;;;;;;;;;;;;;;;;;;;;;;;;;;;;;;;;;;;;;;;;;;;;;;;;;;;;;;;;;;;;;;;;;
  ;;; GNU-Emacs
  ;;; ---------
  ;;; load ~/.gnu-emacs or, if not exists /etc/skel/.gnu-emacs
  ;;; For a description and the settings see /etc/skel/.gnu-emacs
  ;;;   ... for your private ~/.gnu-emacs your are on your one.
  ;;;;;;;;;;;;;;;;;;;;;;;;;;;;;;;;;;;;;;;;;;;;;;;;;;;;;;;;;;;;;;;;;;;;;;;;;
  (if (file-readable-p "~/.gnu-emacs")
      (load "~/.gnu-emacs" nil t)
    (if (file-readable-p "/etc/skel/.gnu-emacs")
	(load "/etc/skel/.gnu-emacs" nil t)))

  ;; Custom Settings
  ;; ===============
  ;; To avoid any trouble with the customization system of GNU emacs
  ;; we set the default file ~/.gnu-emacs-custom
  (setq custom-file "~/.gnu-emacs-custom")
  (load "~/.gnu-emacs-custom" t t)

)
;;;关闭欢迎
(setq inhibit-splash-screen 1)

\end{verbatim}

\subsection{视觉层配置}
\label{sec:orgef0ea4c}

\begin{verbatim}

(put 'set-goal-column 'disabled nil)
(put 'scroll-left 'disabled nil)
(put 'narrow-to-region 'disabled nil)
(put 'narrow-to-page 'disabled nil)
(setq cursor-type 'bar)
;;;(global-linum-mode 1)
;;;(setq inhibit-splash-screen 1)
(set-face-attribute 'default nil :height 120)
;;;主题
(load-theme 'monokai t)
(setq ;; foreground and background
      monokai-foreground     "#ABB2BF"
      monokai-background     "#282C34"
      ;; highlights and comments
      monokai-comments       "#F8F8F0"
      monokai-emphasis       "#282C34"
      monokai-highlight      "#FFB269"
      monokai-highlight-alt  "#66D9EF"
      monokai-highlight-line "#1B1D1E"
      monokai-line-number    "#F8F8F0"
      ;; colours
      monokai-blue           "#61AFEF"
      monokai-cyan           "#56B6C2"
      monokai-green          "#98C379"
      monokai-gray           "#3E4451"
      monokai-violet         "#C678DD"
      monokai-red            "#E06C75"
      monokai-orange         "#D19A66"
      monokai-yellow         "#E5C07B")

\end{verbatim}

\subsection{快捷键配置}
\label{sec:org50b9f8c}

\begin{verbatim}

(defun open-init-file()
  (interactive)
  (find-file "~/.emacs.d/init.el"))
(global-set-key (kbd "<f2>") 'open-init-file)
;;;最近打开文件
(require 'recentf)
(recentf-mode 1)
(setq recentf-max-menu-item 10)
(global-set-key (kbd "C-x C-r") 'recentf-open-files)
(global-set-key "\C-s" 'swiper)
\end{verbatim}

\subsection{常用变量}
\label{sec:org0ab26a7}

\begin{verbatim}

(delete-selection-mode 1)
(global-hungry-delete-mode t)
(fset 'yes-or-no-p 'y-or-n-p)

\end{verbatim}

\subsection{自动补全,缩进}
\label{sec:org0d9b728}

\begin{verbatim}

(global-company-mode 1)
(ivy-mode t)
(setq ivy-use-virtual-buffers t)
(setq enable-recursive-minibuffers t)
(add-hook 'emacs-lisp-mode-hook 'smartparens-mode)
;;括号高亮
(define-advice show-paren-function (:around (fn) fix-show-paren-function)
  "Highlight enclosing parens."
  (cond ((looking-at-p "\\s(") (funcall fn))
	(t (save-excursion
	     (ignore-errors (backward-up-list))
	     (funcall fn)))))

(setq hippie-expand-try-function-list '(try-expand-debbrev
					try-expand-debbrev-all-buffers
					try-expand-debbrev-from-kill
					try-complete-file-name-partially
					try-complete-file-name
					try-expand-all-abbrevs
					try-expand-list
					try-expand-line
					try-complete-lisp-symbol-partially
					try-complete-lisp-symbol))

(defun my-web-mode-indent-setup ()
  (setq web-mode-markup-indent-offset 2) ; web-mode, html tag in html file
  (setq web-mode-css-indent-offset 2)    ; web-mode, css in html file
  (setq web-mode-code-indent-offset 2)   ; web-mode, js code in html file
  )
(add-hook 'web-mode-hook 'my-web-mode-indent-setup)

\end{verbatim}

\subsection{文件操作}
\label{sec:org202c58b}

\begin{verbatim}

(put 'dired-find-alternate-file 'disabled nil)
;; 主动加载 Dired Mode
;; (require 'dired)
;; (defined-key dired-mode-map (kbd "RET") 'dired-find-alternate-file)

;; 延迟加载
(with-eval-after-load 'dired
    (define-key dired-mode-map (kbd "RET") 'dired-find-alternate-file))
(require 'dired-x)
;;;
(setq-default make-backup-files nil)

;;
;;(global-auto-revert-mode 1)


\end{verbatim}
\end{document}
